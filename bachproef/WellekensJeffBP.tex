%===============================================================================
% LaTeX sjabloon voor de bachelorproef toegepaste informatica aan HOGENT
% Meer info op https://github.com/HoGentTIN/latex-hogent-report
%===============================================================================

\documentclass[dutch,dit,thesis]{hogentreport}

% TODO:
% - If necessary, replace the option `dit`' with your own department!
%   Valid entries are dbo, dbt, dgz, dit, dlo, dog, dsa, soa
% - If you write your thesis in English (remark: only possible after getting
%   explicit approval!), remove the option "dutch," or replace with "english".

\usepackage{lipsum} % For blind text, can be removed after adding actual content

%% Pictures to include in the text can be put in the graphics/ folder
\graphicspath{{../graphics/}}

%% For source code highlighting, requires pygments to be installed
%% Compile with the -shell-escape flag!
% \usepackage[section]{minted}
%% If you compile with the make_thesis.{bat,sh} script, use the following
%% import instead:
\usepackage[section,outputdir=../output]{minted}
\usemintedstyle{solarized-light}
\definecolor{bg}{RGB}{253,246,227} %% Set the background color of the codeframe

%% Change this line to edit the line numbering style:
\renewcommand{\theFancyVerbLine}{\ttfamily\scriptsize\arabic{FancyVerbLine}}

%% Macro definition to load external java source files with \javacode{filename}:
\newmintedfile[javacode]{java}{
    bgcolor=bg,
    fontfamily=tt,
    linenos=true,
    numberblanklines=true,
    numbersep=5pt,
    gobble=0,
    framesep=2mm,
    funcnamehighlighting=true,
    tabsize=4,
    obeytabs=false,
    breaklines=true,
    mathescape=false
    samepage=false,
    showspaces=false,
    showtabs =false,
    texcl=false,
}

% Other packages not already included can be imported here

%%---------- Document metadata -------------------------------------------------
% TODO: Replace this with your own information
\author{Jeff Wellekens}
\supervisor{Dhr. B. Vertonghen}
\cosupervisor{Dhr. G. Den Haese}
\title[]%
    {Hoe kan een onvervalsbaar en veilig digitaal tweeling-systeem ontworpen worden dat gebruikmaakt van NFC technologie (DNA-tags), een serverless backend en smart contracts om de authenticiteit en integriteit van producten te garanderen door het gehele proces, van fabricage tot eindgebruiker?}
\academicyear{\advance\year by -1 \the\year--\advance\year by 1 \the\year}
\examperiod{1}
\degreesought{\IfLanguageName{dutch}{Professionele bachelor in de toegepaste informatica}{Bachelor of applied computer science}}
\partialthesis{false} %% To display 'in partial fulfilment'
%\institution{Internshipcompany BVBA.}

%% Add global exceptions to the hyphenation here
\hyphenation{back-slash}

%% The bibliography (style and settings are  found in hogentthesis.cls)
\addbibresource{bachproef.bib}            %% Bibliography file
\addbibresource{../voorstel/voorstel.bib} %% Bibliography research proposal
\defbibheading{bibempty}{}

%% Prevent empty pages for right-handed chapter starts in twoside mode
\renewcommand{\cleardoublepage}{\clearpage}

\renewcommand{\arraystretch}{1.2}

%% Content starts here.
\begin{document}

%---------- Front matter -------------------------------------------------------

\frontmatter

\hypersetup{pageanchor=false} %% Disable page numbering references
%% Render a Dutch outer title page if the main language is English
\IfLanguageName{english}{%
    %% If necessary, information can be changed here
    \degreesought{Professionele Bachelor toegepaste informatica}%
    \begin{otherlanguage}{dutch}%
       \maketitle%
    \end{otherlanguage}%
}{}

%% Generates title page content
\maketitle
\hypersetup{pageanchor=true}

\input{voorwoord}
\input{samenvatting}

%---------- Inhoud, lijst figuren, ... -----------------------------------------

\tableofcontents

% In a list of figures, the complete caption will be included. To prevent this,
% ALWAYS add a short description in the caption!
%
%  \caption[short description]{elaborate description}
%
% If you do, only the short description will be used in the list of figures

\listoffigures

% If you included tables and/or source code listings, uncomment the appropriate
% lines.
%\listoftables
%\listoflistings

% Als je een lijst van afkortingen of termen wil toevoegen, dan hoort die
% hier thuis. Gebruik bijvoorbeeld de ``glossaries'' package.
% https://www.overleaf.com/learn/latex/Glossaries

%---------- Kern ---------------------------------------------------------------

\mainmatter{}

% De eerste hoofdstukken van een bachelorproef zijn meestal een inleiding op
% het onderwerp, literatuurstudie en verantwoording methodologie.
% Aarzel niet om een meer beschrijvende titel aan deze hoofdstukken te geven of
% om bijvoorbeeld de inleiding en/of stand van zaken over meerdere hoofdstukken
% te verspreiden!

\input{inleiding}
\chapter{\IfLanguageName{dutch}{Stand van zaken}{State of the art}}%
\label{ch:stand-van-zaken}

% Tip: Begin elk hoofdstuk met een paragraaf inleiding die beschrijft hoe
% dit hoofdstuk past binnen het geheel van de bachelorproef. Geef in het
% bijzonder aan wat de link is met het vorige en volgende hoofdstuk.
Dit hoofdstuk vormt een gedetailleerde verkenning van de huidige stand van zaken binnen het onderzoeksdomein van digitale tweelingen, NFC-technologie, serverless back-end architecturen en smart contracts, gericht op het waarborgen van de authenticiteit en integriteit van producten gedurende het hele productie- en distributieproces. Deze literatuurstudie dient als een essentiële schakel tussen de inleiding en de verdere ontwikkeling van het onderwerp in deze bachelorproef. We beginnen met een inleiding tot digitale tweelingen en NFC-technologie, waarbij we de basis leggen voor het begrip van de gebruikte technologieën en concepten. Vervolgens onderzoeken we bestaande methoden voor productauthenticatie, inclusief hun beperkingen en uitdagingen, om context te bieden voor het voorgestelde systeem. Daarna richten we ons op de toepassingen van NFC-technologie in productauthenticatie, gevolgd door een bespreking van serverless back-end architecturen en de rol van smart contracts binnen deze context. Ten slotte analyseren we de integratie van deze componenten en behandelen we belangrijke beveiligings- en privacyoverwegingen. Deze inleidende verkenning biedt niet alleen een overzicht van de huidige state-of-the-art, maar legt ook de basis voor de verdere uitwerking van ons voorgestelde systeem voor een onvervalsbaar en veilig digitaal tweeling-systeem.
% Pas na deze inleidende paragraaf komt de eerste sectiehoofding.
\section{\IfLanguageName{dutch}{Introductie in digitale tweelingen en NFC-technologie}{Introduction to digital twins and NFC technology}}
\label{sec:introductie}
\subsection{\IfLanguageName{dutch}{Definitie van digitale tweelingen}{Definition of digital twins}}
Digitale tweelingen vertegenwoordigen een technologische discipline die verschillende domeinen integreert. Op dit moment ontbreekt een uniforme definitie voor digitale tweelingen, aangezien deze definitie onderhevig is aan voortdurende ontwikkeling en evolutie. \autocite{Guo_2022}
Hieronder worden de definities van datatweelingen in diverse contexten en ontwikkelingsfasen weergegeven:
\begin{itemize}
    \item In 2012 definieerde NASA een datatweeling als een uitgebreid multi-fysisch, multi-schaal, probabilistisch simulatiesysteem voor voertuigen of systemen. Dit systeem maakt gebruik van het beste fysieke model om het historische gebruik van apparatuur te beschrijven, met als doel de levensduur van de overeenkomstige fysieke apparatuur te voorspellen \autocite{Guo_2022}.
    \item In 2017 definieerde de Defense Procurement University een datatweeling als een geïntegreerd multifysisch, multischaal, probabilistisch simulatiesysteem dat gebruikmaakt van de best beschikbare modellen, sensorinformatie en invoergegevens om de levensduur, activiteit en prestaties van de corresponderende fysieke tweeling te weerspiegelen en te voorspellen. Dit wordt mogelijk gemaakt door de digitale draad \autocite{Guo_2022}. (Een digitale draad is een digitale representatie van de levenscyclus van een product, van ontwerp tot productie, onderhoud en verder, waardoor een naadloze stroom van gegevens ontstaat die alle aspecten van de levenscyclus met elkaar verbindt. Het doel van een digitale draad is om een volledig en transparant beeld van productiesystemen te bieden, waardoor efficiënte samenwerking en besluitvorming mogelijk zijn in alle stadia van het proces. Digitale draden maken gebruik van verschillende technologieën, waaronder computerondersteund ontwerp (CAD) software, productlevenscyclusbeheer (PLM) systemen en Internet of Things (IoT) sensoren, om gegevens te verzamelen en te delen over workflows. Digitale draadtechnologie optimaliseert traceerbaarheid, waardoor asset-voortgang kan worden gevolgd en ervoor kan worden gezorgd dat alle belanghebbenden gedurende het productieproces op dezelfde pagina zitten. \autocite{China2023})
    \item In 2019 definieerde Stark Damerau een datatweeling als een digitale representatie die de functionele beschrijving omvat van het geselecteerde object, of het nu een product of een servicesysteem is. Deze representatie verkrijgt de attributen, condities en gedragingen van het object door middel van modellen, informatie en gegevens gedurende één of meerdere fasen van de levenscyclus \autocite{Guo_2022}.
    \item In 2020 definieerde een whitepaper, gepubliceerd door het China Institute of Electronic Technology Standardization, een datatweeling als het volledige gebruik van gegevens, zoals het fysieke model, sensorupdates en bedrijfsgeschiedenis. Deze definitie omvat het integreren van multidisciplinaire, multifysische, multischaal en multi-waarschijnlijkheidssimulatieprocessen om het in kaart brengen van de virtuele ruimte te voltooien. Hierdoor wordt het volledige levenscyclusproces van de corresponderende fysieke apparatuur weerspiegeld.
\end{itemize}
In samenvatting kan worden gesteld dat de datatwin een veelzijdige definitie heeft, die alle hightechgebieden moet kunnen omvatten. Fuller suggereert dat kunstmatige intelligentie een integraal onderdeel aan het worden is van de datatwin, waarbij wordt onderzocht waar AI-algoritmen kunnen worden toegepast, wat tevens een toekomstige richting vormt.
\section{\IfLanguageName{dutch}{Huidige methoden voor authenticatie en integriteit van producten}{Current methods for authentication and product integrity}}
\label{sec:huidige-methoden}
\section{\IfLanguageName{dutch}{Toepassingen van NFC-technologie in productauthenticatie}{Applications of NFC technology in product authentication}}
\label{sec:nfc-productauthenticatie}
\section{\IfLanguageName{dutch}{Serverless back-end architectuur}{Serverless back-end architecture}}
\label{sec:serverless-back-end-architectuur}
\section{\IfLanguageName{dutch}{Smart contracts en hun rol in productauthenticatie}{Smart contracts and their role in product authentication}}
\label{sec:smart-contracts-productauthenticatie}
\section{\IfLanguageName{dutch}{Integratie van NFC, serverless back-end en smart contracts}{Integration of NFC, serverless back-end and smart contracts}}
\label{sec:integratie}
\section{\IfLanguageName{dutch}{Beveiliging en privacyoverwegingen}{Security and privacy concerns}}
\label{sec:beveiliging-privacy}
\section{\IfLanguageName{dutch}{Case studies en praktijkvoorbeelden}{Case studies and practical examples}}
\label{sec:case-studies-praktijk}


Dit hoofdstuk bevat je literatuurstudie. De inhoud gaat verder op de inleiding, maar zal het onderwerp van de bachelorproef *diepgaand* uitspitten. De bedoeling is dat de lezer na lezing van dit hoofdstuk helemaal op de hoogte is van de huidige stand van zaken (state-of-the-art) in het onderzoeksdomein. Iemand die niet vertrouwd is met het onderwerp, weet nu voldoende om de rest van het verhaal te kunnen volgen, zonder dat die er nog andere informatie moet over opzoeken \autocite{Pollefliet2011}.

Je verwijst bij elke bewering die je doet, vakterm die je introduceert, enz.\ naar je bronnen. In \LaTeX{} kan dat met het commando \texttt{$\backslash${textcite\{\}}} of \texttt{$\backslash${autocite\{\}}}. Als argument van het commando geef je de ``sleutel'' van een ``record'' in een bibliografische databank in het Bib\LaTeX{}-formaat (een tekstbestand). Als je expliciet naar de auteur verwijst in de zin (narratieve referentie), gebruik je \texttt{$\backslash${}textcite\{\}}. Soms is de auteursnaam niet expliciet een onderdeel van de zin, dan gebruik je \texttt{$\backslash${}autocite\{\}} (referentie tussen haakjes). Dit gebruik je bv.~bij een citaat, of om in het bijschrift van een overgenomen afbeelding, broncode, tabel, enz. te verwijzen naar de bron. In de volgende paragraaf een voorbeeld van elk.

\textcite{Knuth1998} schreef een van de standaardwerken over sorteer- en zoekalgoritmen. Experten zijn het erover eens dat cloud computing een interessante opportuniteit vormen, zowel voor gebruikers als voor dienstverleners op vlak van informatietechnologie~\autocite{Creeger2009}.

Let er ook op: het \texttt{cite}-commando voor de punt, dus binnen de zin. Je verwijst meteen naar een bron in de eerste zin die erop gebaseerd is, dus niet pas op het einde van een paragraaf.

% \lipsum[7-20]

\input{methodologie}

% Voeg hier je eigen hoofdstukken toe die de ``corpus'' van je bachelorproef
% vormen. De structuur en titels hangen af van je eigen onderzoek. Je kan bv.
% elke fase in je onderzoek in een apart hoofdstuk bespreken.

%\input{...}
%\input{...}
%...

\input{conclusie}

%---------- Bijlagen -----------------------------------------------------------

\appendix

\chapter{Onderzoeksvoorstel}

Het onderwerp van deze bachelorproef is gebaseerd op een onderzoeksvoorstel dat vooraf werd beoordeeld door de promotor. Dat voorstel is opgenomen in deze bijlage.

%% TODO: 
%\section*{Samenvatting}

% Kopieer en plak hier de samenvatting (abstract) van je onderzoeksvoorstel.

% Verwijzing naar het bestand met de inhoud van het onderzoeksvoorstel
\input{../voorstel/voorstel-inhoud}

%%---------- Andere bijlagen --------------------------------------------------
% TODO: Voeg hier eventuele andere bijlagen toe. Bv. als je deze BP voor de
% tweede keer indient, een overzicht van de verbeteringen t.o.v. het origineel.
%\input{...}

%%---------- Backmatter, referentielijst ---------------------------------------

\backmatter{}

\setlength\bibitemsep{2pt} %% Add Some space between the bibliograpy entries
\printbibliography[heading=bibintoc]

\end{document}
